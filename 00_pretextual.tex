% Isto é um exemplo de Folha de aprovação, elemento obrigatório da NBR
% 14724/2011 (seção 4.2.1.3). Você pode utilizar este modelo até a aprovação
% do trabalho. Após isso, substitua todo o conteúdo deste arquivo por uma
% imagem da página assinada pela banca com o comando abaixo:
%
% \includepdf{folhadeaprovacao_final.pdf}
%
\begin{folhadeaprovacao}

  \begin{center}
    {\ABNTEXchapterfont\large\imprimirautor}

    \vspace*{\fill}\vspace*{\fill}
    {\ABNTEXchapterfont\bfseries\Large\imprimirtitulo}
    \vspace*{\fill}
    
    \hspace{.45\textwidth}
    \begin{minipage}{.5\textwidth}
        \imprimirpreambulo
    \end{minipage}%
    \vspace*{\fill}
   \end{center}
    
   \imprimirlocal, \today
   % Caso prefira fixar a data, basta mudar o comando \today pela data,
   %   como no exemplo abaixo:
   %Trabalho aprovado. \imprimirlocal, 7 de setembro de 1822:

   \assinatura{\textbf{\imprimirorientador} \\ Orientador} 
   \assinatura{\textbf{Prof. Rafael Sachetto Oliveira} \\ Convidado 1}
   \assinatura{\textbf{Prof. Marcelo Lobosco} \\ Convidado 2} 
   \assinatura{\textbf{Profa. Bárbara de Melo Quintela} \\ Convidado 3}   
      
   \begin{center}
    \vspace*{0.5cm}
    {\large\imprimirlocal}
    \par
    {\large\imprimirdata}
    \vspace*{1cm}
  \end{center}
  
\end{folhadeaprovacao}

% % ---------------------------------------------------------------------------------------------
% % ---------------------------------------------------------------------------------Dedicatória-
% % ---------------------------------------------------------------------------------------------
% \begin{dedicatoria}
% 	\vspace*{\fill}
% 	\centering
% 	\noindent
% 	\textit{ Este trabalho é dedicado às crianças adultas que,\\
% 		quando pequenas, sonharam em se tornar cientistas.} \vspace*{\fill}
% \end{dedicatoria}

% ---------------------------------------------------------------------------------------------
% ------------------------------------------------------------------------------Agradecimentos-
% ---------------------------------------------------------------------------------------------
\begin{agradecimentos}
Gostaria de agradecer e dedicar este trabalho:

\noindent
Aos meus pais, pelas oportunidades que me deram, pelo apoio nas escolhas que fiz para chegar aqui, pelo companheirismo e pelo amor incondicional que me proporcionaram.

\noindent
Ao meu orientador, Alexandre Pigozzo, que me introduziu ao meio científico, ao tema deste trabalho e me auxiliou não somente neste trabalho, mas na minha trajetória nesta universidade.

\noindent
Aos servidores do DCOMP-UFSJ, que mantém um alto nível de educação e demonstram respeito e empatia pelos docentes do curso. Em especial, um agradecimento aos servidores(as) Carolina Xavier, Leonardo Rocha, Marcos Laia, Michelli Loureiro e Douglas Dinalli.

\noindent
Aos meus amigos de Guarapari, que sempre estiveram ao meu lado.

\noindent
À minha namorada, Keina Takashiba, minha melhor amiga, que me ajuda a me organizar e me apoia em tudo que for.

\noindent
Aos amigos que fiz aqui, na UFSJ, Arthur Gabriel Matos, Ana Clara Medina, Lucas Grandolpho e Wesley Guimarães, meus grandes companheiros de trabalhos.

\noindent
À FAPEMIG, que apoiou este e os projetos anteriores.

\noindent
À todos aqueles que apoiam, defendem e lutam por uma educação gratuita e de qualidade para todos.
	
\end{agradecimentos}

% ---------------------------------------------------------------------------------------------
% ------------------------------------------------------------------------------------Epígrafe-
% ---------------------------------------------------------------------------------------------
\begin{epigrafe}
	\vspace*{\fill}
	\begin{flushright}

		
	\end{flushright}
\end{epigrafe}
% ---

% ---------------------------------------------------------------------------------------------
% -------------------------------------------------------------------------resumo em português-
% ---------------------------------------------------------------------------------------------
\begin{resumo}
	
	% TODO: Está exatamente igual ao trabalho de AC, com exceção da nomenclatura. Devemos mudar?
	
	\vspace{\onelineskip}
	
	\noindent
	\textbf{Palavras-chaves}: modelagem, simulação, modelos computacionais, software \textit{open source}, interface gráfica do usuário.
	
	As áreas de Modelagem Matemática e Computacional têm se tornado cada vez mais importantes no mundo atual, no qual estudos científicos devem trazer resultados cada vez mais rápidos. Os modelos matemáticos e computacionais surgem como ferramentas poderosas no estudo e compreensão de sistemas complexos e que podem ser usados por pesquisadores de diversas áreas diferentes. Frequentemente, os modelos comumente requerem extensivo conhecimento matemático para serem criados, o que resulta em uma grande barreira de entrada para cientistas sem formação relacionada diretamente com a matemática, como biólogos por exemplo, e estudantes iniciando carreiras acadêmicas. Apesar de existirem softwares que auxiliam o desenvolvimento de modelos computacionais, estes frequentemente apresentam interfaces muito complexas na busca para se tornarem genéricos o suficiente e fornecerem muitos recursos. Neste trabalho, é apresentado o software que foi desenvolvido para construir e simular modelos de Equações Diferenciais Ordinárias. O objetivo foi desenvolver um software simples, fácil de usar e de estender com novas funcionalidades. Através da interface gráfica, o usuário pode construir um modelo utilizando um editor baseado em nós, realizar simulações e gerar o código que implementa o modelo. O software poderá ser utilizado na pesquisa e no processo ensino-aprendizagem de modelagem computacional. 
	
\end{resumo}

% ---------------------------------------------------------------------------------------------
% ----------------------------------------------------------------------------resumo em inglês-
% ---------------------------------------------------------------------------------------------
\begin{resumo}[Abstract]
	\begin{otherlanguage*}{english}
    
    
		\vspace{\onelineskip}
		
		\noindent 
		\textbf{Key-words}: modelling, simulation, automata, computational models, mathematical models, graphical interface software.
		
		The areas of Mathematical and Computational Modeling have become increasingly important in today's world, in which scientific studies must bring increasingly faster results. Mathematical and computational models emerge as powerful tools for studying and understanding complex systems and can be used by researchers from several different areas. Often, models commonly require extensive mathematical knowledge to be created, which results in a high entry barrier for scientists without mathematical training, such as biologists, for example, and students starting academic careers. Although there is software that helps the development of computational models, they often present very complex interfaces in order to become generic enough and provide many resources. In this work, the software that was developed to build and simulate models of Ordinary Differential Equations is presented. The objective was to develop a simple software, easy to use and extend with new features. Through the graphical user interface, the user can build a model using a node-based editor, perform simulations and generate the code that implements the model. The software can be used in research and in the teaching-learning process of computational modeling.
	\end{otherlanguage*}
\end{resumo}
